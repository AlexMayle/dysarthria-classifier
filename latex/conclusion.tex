\section{Conclusion}
This paper investigates the effectiveness of LSTM networks in the classification of dysarthria among both afflicted and healthy Mandarin speakers. When presented with a single syllable pronunciation, we found that single and double layer, one-directional LSTM networks slightly outperform their bidirectional single layer counterpart. Further, the double-layer LSTM regularized with dropout between layers exhibit an improvement in the rate of false negatives. While these methods are not yet practical as a standalone medical test, they do suggest that LSTM networks may provide a fruitful avenue for the realization of autonomous dysarthria classification.